\documentclass[aspectratio=169, table]{beamer}
\usepackage[utf8]{inputenc}
\usepackage[T1]{fontenc}
\usepackage{graphicx}
\usepackage{fontspec}
\usepackage{xcolor}
\usepackage{tcolorbox}
\usepackage{listings} % Add the listings package
\usepackage{hyperref} % Add the hyperref package

\lstdefinelanguage{JavaScript}{
    keywords={function, var, let, const, if, else, for, while, return, true, false},
    keywordstyle=\color{blue}\bfseries,
    ndkeywords={class, export, boolean, throw, implements, import, this},
    ndkeywordstyle=\color{orange}\bfseries,
    identifierstyle=\color{black},
    sensitive=false,
    comment=[l]{//},
    morecomment=[s]{/*}{*/},
    commentstyle=\color{gray}\ttfamily,
    stringstyle=\color{green}\ttfamily,
}

\lstset{
    breaklines=true,
    language=JavaScript,
    % ... (other style settings)
}

\lstdefinelanguage{PHP}{
    keywords={class, function, echo, if, else, foreach, for, while, return},
    keywordstyle=\color{blue}\bfseries,
    ndkeywords={public, private, protected, static},
    ndkeywordstyle=\color{purple}\bfseries,
    identifierstyle=\color{black},
    sensitive=false,
    comment=[l]{//},
    morecomment=[s]{/*}{*/},
    commentstyle=\color{gray}\ttfamily,
    stringstyle=\color{green}\ttfamily,
}

\lstset{
    breaklines=true,
    language=PHP,
    % ... (other style settings)
}

\setsansfont[
  ItalicFont=fonts/TitilliumWeb-Italic.ttf,
  BoldFont=fonts/TitilliumWeb-Bold.ttf,
  BoldItalicFont=fonts/TitilliumWeb-BoldItalic.ttf,
]{TitilliumWeb-Regular.ttf}

\subtitle{IF120203-Web Programming}
\title{\Huge {\textbf{12: \\Database}}}
\date[Serial]{\scriptsize {PRU/SPMI/FR-BM-18/0222}}
\author[Pradita]{\small {\textbf{PRADITA UNIVERSITY}}}

\usetheme{Pradita}

\begin{document}
\begin{frame}
    \titlepage
\end{frame}

\begin{frame}{Goals - Database}
    \vskip1cm
    \begin{itemize}
        \item Understand the fundamentals of databases and their role in web development.
        \item Learn about relational databases and their importance in organizing structured data.
        \item Explore database management systems (DBMS) and their various types.
        \item Gain insight into the use of SQL (Structured Query Language) for database interactions.
        \item Learn about creating, reading, updating, and deleting (CRUD) operations on databases.
        \item Explore the concept of database migrations for version control and collaboration.
        \item Understand the significance of database normalization for efficient data storage.
        \item Gain hands-on experience by integrating databases with web applications.
    \end{itemize}
\end{frame}

\begin{frame}{Introduction to Database}
    \vskip1cm
    \begin{itemize}
        \item A database is a structured collection of data that is organized and managed to provide efficient retrieval and manipulation.
        \item Databases play a crucial role in web development by providing a way to store, manage, and retrieve data for applications.
        \item Databases are used to store various types of information, including user data, product information, orders, and more.
        \item Web applications often require efficient and reliable data storage to provide a seamless user experience.
        \item Different types of databases exist, such as relational databases, NoSQL databases, and in-memory databases.
        \item The choice of a database type depends on factors like the nature of data, scalability, and performance requirements.
    \end{itemize}
\end{frame}

\begin{frame}{Migration Laravel}
    \vskip1cm
    \begin{itemize}
        \item Migration is a version control for the database schema
        \item In Laravel Project CMD: php artisan make:migration create\_converters\_table
        \item Running Migration: php artisan migrate (if refresh is : php artisan migrate:refresh)
        \item Code in Next Slide
    \end{itemize}
\end{frame}

\begin{frame}[fragile]
 \frametitle{Migration Laravel}
 \vskip1cm
 \begin{center}
  \includegraphics[width=0.6\textwidth]{classFiles/pertemuan-12-migrate.png}
 \end{center}
\end{frame}

\begin{frame}{Seeder Laravel}
    \vskip1cm
    \begin{itemize}
        \item Seeder allow you easily populate your database with sample or default data.
        \item In Laravel Project CMD: php artisan make:seeder ConverterSeeder
        \item Code in Next Slide
    \end{itemize}
\end{frame}

\begin{frame}[fragile]
 \frametitle{Seeder Laravel}
 \vskip1cm
 \begin{center}
  \includegraphics[width=0.6\textwidth]{classFiles/pertemuan-12-seeder.png}
 \end{center}
\end{frame}

\begin{frame}{Models Laravel}
    \vskip1cm
    \begin{itemize}
        \item Models provide an abstraction layer that simplifies database interactions and promotes good coding practices.
        \item In Laravel Project CMD: php artisan make:model Converter
        \item Old Converter model change the name to Converter\_.php or remove it
        \item Code in Next Slide
    \end{itemize}
\end{frame}

\begin{frame}[fragile]
 \frametitle{Model Laravel}
 \vskip1cm
 \begin{center}
  \includegraphics[width=0.6\textwidth]{classFiles/pertemuan-12-model.png}
 \end{center}
\end{frame}

\begin{frame}{Views Laravel}
    \vskip1cm
    \begin{itemize}
        \item Create add-converter.blade.php, edit-converter.blade.php
        \item 
    \end{itemize}
\end{frame}

\begin{frame}[fragile]
 \frametitle{add-converter.blade.php Part 1}
 \vskip1cm
 \begin{center}
  \includegraphics[width=0.6\textwidth]{classFiles/pertemuan-12-view-part-1.png}
 \end{center}
\end{frame}

\begin{frame}[fragile]
 \frametitle{add-converter.blade.php Part 2}
 \vskip1cm
 \begin{center}
  \includegraphics[width=0.6\textwidth]{classFiles/pertemuan-12-view-part-2.png}
 \end{center}
\end{frame}

\begin{frame}[fragile]
 \frametitle{edit-converter.blade.php Part 1}
 \vskip1cm
 \begin{center}
  \includegraphics[width=0.6\textwidth]{classFiles/pertemuan-12-view-part-3.png}
 \end{center}
\end{frame}

\begin{frame}[fragile]
 \frametitle{edit-converter.blade.php Part 2}
 \vskip1cm
 \begin{center}
  \includegraphics[width=0.6\textwidth]{classFiles/pertemuan-12-view-part-4.png}
 \end{center}
\end{frame}

\begin{frame}[fragile]
 \frametitle{ConverterController.php Part 1}
 \vskip1cm
 \begin{center}
  \includegraphics[width=0.6\textwidth]{classFiles/pertemuan-12-controller-part-1.png}
 \end{center}
\end{frame}

\begin{frame}[fragile]
 \frametitle{ConverterController.php Part 2}
 \vskip1cm
 \begin{center}
  \includegraphics[width=0.6\textwidth]{classFiles/pertemuan-12-controller-part-2.png}
 \end{center}
\end{frame}

\begin{frame}[fragile]
 \frametitle{ConverterController.php Part 3}
 \vskip1cm
 \begin{center}
  \includegraphics[width=0.6\textwidth]{classFiles/pertemuan-12-controller-part-3.png}
 \end{center}
\end{frame}

\begin{frame}[fragile]
 \frametitle{ConverterController.php Part 4}
 \vskip1cm
 \begin{center}
  \includegraphics[width=0.6\textwidth]{classFiles/pertemuan-12-controller-part-4.png}
 \end{center}
\end{frame}

\begin{frame}[fragile]
 \frametitle{ConverterController.php Part 5}
 \vskip1cm
 \begin{center}
  \includegraphics[width=0.6\textwidth]{classFiles/pertemuan-12-controller-part-5.png}
 \end{center}
\end{frame}

\begin{frame4}
    \frametitle{Thank You}
\end{frame4}

\end{document}
