\documentclass[aspectratio=169, table]{beamer}
\usepackage[utf8]{inputenc}
\usepackage[T1]{fontenc}
\usepackage{graphicx}
\usepackage{fontspec}
\usepackage{xcolor}
\usepackage{tcolorbox}
\usepackage{listings} % Add the listings package
\usepackage{hyperref} % Add the hyperref package

\lstdefinelanguage{JavaScript}{
    keywords={function, var, let, const, if, else, for, while, return, true, false},
    keywordstyle=\color{blue}\bfseries,
    ndkeywords={class, export, boolean, throw, implements, import, this},
    ndkeywordstyle=\color{orange}\bfseries,
    identifierstyle=\color{black},
    sensitive=false,
    comment=[l]{//},
    morecomment=[s]{/*}{*/},
    commentstyle=\color{gray}\ttfamily,
    stringstyle=\color{green}\ttfamily,
}

\lstset{
    breaklines=true,
    language=JavaScript,
    % ... (other style settings)
}

\lstdefinelanguage{PHP}{
    keywords={class, function, echo, if, else, foreach, for, while, return},
    keywordstyle=\color{blue}\bfseries,
    ndkeywords={public, private, protected, static},
    ndkeywordstyle=\color{purple}\bfseries,
    identifierstyle=\color{black},
    sensitive=false,
    comment=[l]{//},
    morecomment=[s]{/*}{*/},
    commentstyle=\color{gray}\ttfamily,
    stringstyle=\color{green}\ttfamily,
}

\lstset{
    breaklines=true,
    language=PHP,
    % ... (other style settings)
}

\setsansfont[
  ItalicFont=fonts/TitilliumWeb-Italic.ttf,
  BoldFont=fonts/TitilliumWeb-Bold.ttf,
  BoldItalicFont=fonts/TitilliumWeb-BoldItalic.ttf,
]{TitilliumWeb-Regular.ttf}

\subtitle{IF120203-Web Programming}
\title{\Huge {\textbf{08: \\Web Framework Laravel}}}
\date[Serial]{\scriptsize {PRU/SPMI/FR-BM-18/0222}}
\author[Pradita]{\small {\textbf{PRADITA UNIVERSITY}}}

\usetheme{Pradita}

\begin{document}
\begin{frame}
    \titlepage
\end{frame}

\begin{frame}{Goals - Using Laravel}
    \begin{itemize}
        \item To learn about Laravel, a powerful and elegant PHP framework for web application development
        \item To understand the key concepts of Laravel, including routing, controllers, views, and models
        \item To explore Laravel's built-in features such as authentication, database migration, and templating
        \item To build dynamic and interactive web applications efficiently using Laravel's expressive syntax and developer-friendly tools
        \item To apply best practices in web development by utilizing Laravel's conventions and security mechanisms
    \end{itemize}
\end{frame}

\begin{frame}{Introduction to Laravel}
    \begin{itemize}
        \item Laravel is a popular open-source PHP framework known for its elegant syntax and robust features.
        \item It follows the Model-View-Controller (MVC) architectural pattern, separating application logic from presentation.
        \item Laravel provides a wide range of tools and libraries for tasks like routing, database interaction, and user authentication.
        \item Artisan, Laravel's command-line tool, simplifies common development tasks such as creating models, controllers, and migrations.
        \item Laravel promotes code reusability and maintainability through features like middleware and dependency injection.
    \end{itemize}
\end{frame}

\begin{frame}{Prerequisite}
    \vskip-1cm
    \begin{itemize}
        \item Good at OOP
        \item Know about MVC Architecture
        \item HTML, CSS, Javascript for Frontend
        \item PHP
    \end{itemize}
\end{frame}

\begin{frame}[fragile]
    \frametitle{Handling Views in Laravel}
    \begin{itemize}
        \item All views are recommended to use the format \texttt{{name}}.blade.php.
        \item This convention is followed because Laravel uses the Blade template engine.
        \item Blade provides features like template inheritance, control structures, and more to simplify view rendering.
        \item With Blade, you can create dynamic and reusable components for your application's frontend.
        \item Blade templates are compiled into regular PHP code for better performance.
        \item Blade's syntax is intuitive and expressive, making it easy to work with complex views.
    \end{itemize}
\end{frame}

\begin{frame}[fragile]
    \frametitle{Handling Controllers in Laravel Part 1}
    \begin{itemize}
        \item Controllers play a crucial role in handling the application's logic and responding to user requests.
        \item In Laravel, controllers are used to group related request handling logic into a single class.
        \item Controllers help to achieve a separation of concerns by keeping the application's routing logic separate from the actual request processing.
    \end{itemize}
\end{frame}

\begin{frame}[fragile]
    \frametitle{Handling Controllers in Laravel Part 2}
    \begin{itemize}
        \item You can define multiple methods within a controller to handle different HTTP verbs and actions.
        \item Laravel provides a powerful command-line tool, Artisan, to generate controllers effortlessly.
        \item Controllers facilitate the reuse of code and enhance maintainability by keeping code organized and structured.
        \item By returning views or data from controllers, you control what content is displayed to the user.
        \item Controllers can also handle tasks like validation, authentication, and more, providing a central point for such functionalities.
    \end{itemize}
\end{frame}

\begin{frame}[fragile]
    \frametitle{Database Configuration in Laravel Part 1}
    \begin{itemize}
        \item Laravel provides a powerful database abstraction layer that makes working with databases easy and efficient.
        \item To configure the database connection, you need to open the `config/database.php` file and `.env file`.
        \item In this file, you can set various database connections like MySQL, PostgreSQL, SQLite, and more.
        \item The configuration options include the driver, host, port, database name, username, and password.
    \end{itemize}
\end{frame}

\begin{frame}[fragile]
    \frametitle{Database Configuration in Laravel Part 2}
    \begin{itemize}
        \item Laravel supports query building and the Eloquent ORM (Object-Relational Mapping) for interacting with the database.
        \item Eloquent provides an elegant syntax for creating, querying, and manipulating database records using PHP classes.
        \item You can define models to represent database tables and use them to perform various database operations.
        \item By utilizing migrations, you can version-control your database schema and keep it consistent across different environments.
        \item Laravel's database configuration and features make database management and interaction seamless and efficient.
    \end{itemize}
\end{frame}


\begin{frame}[fragile]
    \frametitle{Using MVC make Controller}
    \begin{itemize}
        \item Install Composer for using Laravel with this command
        \item composer create-project laravel/laravel project-name (add this command in CLI)
        \item php artisan make:controller ConverterController --resource (add this command in CLI after create and cd to the project folder)
        \item --resource is to have a basic CRUD
        \item learn more https://laravel.com/docs/10.x
    \end{itemize}
\end{frame}

\begin{frame}[fragile]
    \frametitle{ConverterController.php}
    \begin{lstlisting}[language=PHP]
public function index()
    {
        $data['title'] = 'Currency Converter';
        return view('converter', $data); //digunakan untuk menampilkan view yang ada di folder views
    }
    \end{lstlisting}
\end{frame}

\begin{frame}[fragile]
    \frametitle{Public Folder in Laravel}
    \begin{itemize}
        \item In Laravel, the `public` folder serves as the document root for your application when accessed through a web browser.
        \item All publicly accessible files, such as images, JavaScript, CSS, and other assets, should be placed in this folder.
        \item The contents of the `public` folder are exposed to the internet, making it the ideal location for assets that need to be directly accessed by users.
    \end{itemize}
\end{frame}

\begin{frame}[fragile]
    \frametitle{Public Folder in Laravel}
    \begin{itemize}
        \item Files in the `public` folder are accessible using the URL of your application followed by the relative path of the file.
        \item For example, if you have an image named `logo.png` in the `public/images` directory, it can be accessed using `http://yourdomain.com/images/logo.png`.
        \item Laravel's `asset()` helper function can be used to generate URLs for assets in the `public` folder, making it easy to reference them in your views and templates.
        \item Remember to place only public assets in the `public` folder, as files here can be directly accessed by anyone with the appropriate URL.
    \end{itemize}
\end{frame}

\begin{frame}[fragile]
    \frametitle{Public Folder in Laravel}
    \begin{itemize}
        \item In meeting 8, we are still using an external Currency API.
        \item Please using meeting 5 assets place it in public folder (css, js, image)
        \item Laravel's `asset()` helper function can be used to generate URLs for assets in the `public` folder, making it easy to reference them in your views and templates.
        \item Remember to place only public assets in the `public` folder, as files here can be directly accessed by anyone with the appropriate URL.
    \end{itemize}
\end{frame}


\begin{frame4}
    \frametitle{Thank You}
\end{frame4}

\end{document}
