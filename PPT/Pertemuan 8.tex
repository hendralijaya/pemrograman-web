\documentclass[aspectratio=169, table]{beamer}
\usepackage[utf8]{inputenc}
\usepackage[T1]{fontenc}
\usepackage{graphicx}
\usepackage{fontspec}
\usepackage{xcolor}
\usepackage{tcolorbox}
\usepackage{listings} % Add the listings package
\usepackage{hyperref} % Add the hyperref package

\lstdefinelanguage{JavaScript}{
    keywords={function, var, let, const, if, else, for, while, return, true, false},
    keywordstyle=\color{blue}\bfseries,
    ndkeywords={class, export, boolean, throw, implements, import, this},
    ndkeywordstyle=\color{orange}\bfseries,
    identifierstyle=\color{black},
    sensitive=false,
    comment=[l]{//},
    morecomment=[s]{/*}{*/},
    commentstyle=\color{gray}\ttfamily,
    stringstyle=\color{green}\ttfamily,
}

\lstset{
    breaklines=true,
    language=JavaScript,
    % ... (other style settings)
}

\lstdefinelanguage{PHP}{
    keywords={class, function, echo, if, else, foreach, for, while, return},
    keywordstyle=\color{blue}\bfseries,
    ndkeywords={public, private, protected, static},
    ndkeywordstyle=\color{purple}\bfseries,
    identifierstyle=\color{black},
    sensitive=false,
    comment=[l]{//},
    morecomment=[s]{/*}{*/},
    commentstyle=\color{gray}\ttfamily,
    stringstyle=\color{green}\ttfamily,
}

\lstset{
    breaklines=true,
    language=PHP,
    % ... (other style settings)
}

\setsansfont[
  ItalicFont=fonts/TitilliumWeb-Italic.ttf,
  BoldFont=fonts/TitilliumWeb-Bold.ttf,
  BoldItalicFont=fonts/TitilliumWeb-BoldItalic.ttf,
]{TitilliumWeb-Regular.ttf}

\subtitle{IF120203-Web Programming}
\title{\Huge {\textbf{08: \\Web Framework Laravel}}}
\date[Serial]{\scriptsize {PRU/SPMI/FR-BM-18/0222}}
\author[Pradita]{\small {\textbf{PRADITA UNIVERSITY}}}

\usetheme{Pradita}

\begin{document}
\begin{frame}
    \titlepage
\end{frame}

\begin{frame}{Goals - Using Laravel}
    \vskip-1cm
    \begin{itemize}
        \item To learn about Laravel, a powerful and elegant PHP framework for web application development
        \item To understand the key concepts of Laravel, including routing, controllers, views, and models
        \item To explore Laravel's built-in features such as authentication, database migration, and templating
        \item To build dynamic and interactive web applications efficiently using Laravel's expressive syntax and developer-friendly tools
        \item To apply best practices in web development by utilizing Laravel's conventions and security mechanisms
    \end{itemize}
\end{frame}

\begin{frame}{Introduction to Laravel}
    \vskip-0.5cm
    \begin{itemize}
        \item Laravel is a popular open-source PHP framework known for its elegant syntax and robust features.
        \item It follows the Model-View-Controller (MVC) architectural pattern, separating application logic from presentation.
        \item Laravel provides a wide range of tools and libraries for tasks like routing, database interaction, and user authentication.
        \item Artisan, Laravel's command-line tool, simplifies common development tasks such as creating models, controllers, and migrations.
        \item Laravel promotes code reusability and maintainability through features like middleware and dependency injection.
    \end{itemize}
\end{frame}

\begin{frame}{Prerequisite}
    \vskip-1cm
    \begin{itemize}
        \item Good at OOP
        \item Know about MVC Architecture
        \item HTML, CSS, Javascript for Frontend
        \item PHP
    \end{itemize}
\end{frame}

\begin{frame}[fragile]
    \frametitle{Handling Views in Laravel}
    \begin{itemize}
        \item All views are recommended to use the format \texttt{{name}}.blade.php.
        \item This convention is followed because Laravel uses the Blade template engine.
        \item Blade provides features like template inheritance, control structures, and more to simplify view rendering.
        \item With Blade, you can create dynamic and reusable components for your application's frontend.
        \item Blade templates are compiled into regular PHP code for better performance.
        \item Blade's syntax is intuitive and expressive, making it easy to work with complex views.
    \end{itemize}
\end{frame}

\begin{frame}[fragile]
    \frametitle{Handling Controllers in Laravel Part 1}
    \begin{itemize}
        \item Controllers play a crucial role in handling the application's logic and responding to user requests.
        \item In Laravel, controllers are used to group related request handling logic into a single class.
        \item Controllers help to achieve a separation of concerns by keeping the application's routing logic separate from the actual request processing.
    \end{itemize}
\end{frame}

\begin{frame}[fragile]
    \frametitle{Handling Controllers in Laravel Part 2}
    \begin{itemize}
        \item You can define multiple methods within a controller to handle different HTTP verbs and actions.
        \item Laravel provides a powerful command-line tool, Artisan, to generate controllers effortlessly.
        \item Controllers facilitate the reuse of code and enhance maintainability by keeping code organized and structured.
        \item By returning views or data from controllers, you control what content is displayed to the user.
        \item Controllers can also handle tasks like validation, authentication, and more, providing a central point for such functionalities.
    \end{itemize}
\end{frame}

\begin{frame}[fragile]
    \frametitle{Using MVC make Controller}
    \begin{itemize}
        \item Install Composer for using Laravel with this command
        \item composer create-project laravel/laravel project-name (add this command in CLI)
        \item php artisan make:controller ConverterController --resource (add this command in CLI after create and cd to the project folder)
        \item --resource is to have a basic CRUD
        \item learn more https://laravel.com/docs/10.x
    \end{itemize}
\end{frame}

\begin{frame}[fragile]
    \frametitle{ConverterController.php}
    \begin{lstlisting}[language=PHP]
public function index()
    {
        $data['title'] = 'Currency Converter';
        return view('converter', $data);
    }
    \end{lstlisting}
\end{frame}

\begin{frame}[fragile]
    \frametitle{JQuery Convert Button - Part 1}
    \begin{lstlisting}[language=JavaScript]
$("#convert").click(function (e) {
    e.preventDefault(); // Prevent the form from submitting
    const fromSelect = $("#from");
    const toSelect = $("#to");
    const amount = $("#amount").val();
    const fromCurrency = fromSelect.val();
    const toCurrency = toSelect.val();
    \end{lstlisting}
\end{frame}

\begin{frame}[fragile]
    \frametitle{JQuery Convert Button - Part 2}
    \begin{lstlisting}[language=JavaScript]
$.ajax({
      url: "converter.php",
      type: "POST",
      data: {
        amount: amount,
        from: fromCurrency,
        to: toCurrency,
      },
    \end{lstlisting}
\end{frame}

\begin{frame}[fragile]
    \frametitle{JQuery Convert Button - Part 3}
    \begin{lstlisting}[language=JavaScript]
success: function (result) {
        const rate = result.rate;
        const convertedAmount = (rate * amount).toLocaleString(undefined, {
          minimumFractionDigits: 2,
          maximumFractionDigits: 2,
        });
    \end{lstlisting}
\end{frame}

\begin{frame}[fragile]
    \frametitle{JQuery Convert Button - Part 4}
    \begin{lstlisting}[language=JavaScript]
let fromCurrencySymbol;
        if (fromCurrency === "IDR") {
          fromCurrencySymbol = "Rp. ";
        } else if (fromCurrency === "USD") {
          fromCurrencySymbol = "$ ";
        } else if (fromCurrency === "EUR") {
          fromCurrencySymbol = "€ ";
        } else if (fromCurrency === "GBP") {
          fromCurrencySymbol = "£ ";
        }
    \end{lstlisting}
\end{frame}

\begin{frame}[fragile]
    \frametitle{JQuery Convert Button - Part 5}
    \begin{lstlisting}[language=JavaScript]
let toCurrencySymbol;
        if (toCurrency === "IDR") {
          toCurrencySymbol = "Rp. ";
        } else if (toCurrency === "USD") {
          toCurrencySymbol = "$ ";
        } else if (toCurrency === "EUR") {
          toCurrencySymbol = "€ ";
        } else if (toCurrency === "GBP") {
          toCurrencySymbol = "£ ";
        }
    \end{lstlisting}
\end{frame}

\begin{frame}[fragile]
    \frametitle{JQuery Convert Button - Part 6}
    \begin{lstlisting}[language=JavaScript]
$("#from-result").text(fromCurrencySymbol + amount);
        $("#from-currency").text(fromCurrency);
        $("#to-result").text(toCurrencySymbol + convertedAmount);
        $("#to-currency").text(toCurrency);
        $("#exchange-rate").text(`1 ${fromCurrency} = ${rate} ${toCurrency}`);
      },
    \end{lstlisting}
\end{frame}

\begin{frame}[fragile]
    \frametitle{JQuery Convert Button - Part 7}
    \begin{lstlisting}[language=JavaScript]
error: function (error) {
        console.error(error);
      },
    });
  });
});
    \end{lstlisting}
\end{frame}

\begin{frame}{Change HTML to PHP}
    \vskip-1cm
    \begin{itemize}
        \item change converter.html to converter.php
        \item adding PHP script on top of the file
    \end{itemize}
\end{frame}

\begin{frame}[fragile]
    \frametitle{Adding PHP Script on Top - Part 1}
    \begin{lstlisting}[language=PHP]
<?php
function convertRate($fromCurrency, $toCurrency)
{
  $apiKey = "fca_live_iq1WvbaON67X9adHTaJiqEszDhP6jtDz7IouUbWv";
  $url = "https://api.freecurrencyapi.com/v1/latest?apikey=$apiKey&currencies=$toCurrency&base_currency=$fromCurrency";

  $ch = curl_init($url);
    \end{lstlisting}
\end{frame}

\begin{frame}[fragile]
    \frametitle{Adding PHP Script on Top - Part 2}
    \begin{lstlisting}[language=PHP]
  curl_setopt($ch, CURLOPT_RETURNTRANSFER, true);
  $response = curl_exec($ch);
  curl_close($ch);
  $data = json_decode($response, true);
  return $data['data'][$toCurrency];
}
    \end{lstlisting}
\end{frame}

\begin{frame}[fragile]
    \frametitle{Adding PHP Script on Top - Part 3}
    \begin{lstlisting}[language=PHP]
if ($_SERVER['REQUEST_METHOD'] == 'POST') {
  $amount = $_POST['amount'];
  $fromCurrency = $_POST['from'];
  $toCurrency = $_POST['to'];
  $rate = convertRate($fromCurrency, $toCurrency);
    \end{lstlisting}
\end{frame}

\begin{frame}[fragile]
    \frametitle{Adding PHP Script on Top - Part 4}
    \begin{lstlisting}[language=PHP]
$response = array(
    'rate' => $rate
  );

  header('Content-Type: application/json');
  echo json_encode($response);
  exit;
}
?>
    \end{lstlisting}
\end{frame}

\begin{frame4}
    \frametitle{Thank You}
\end{frame4}

\end{document}
