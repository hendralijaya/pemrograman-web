\documentclass[aspectratio=169, table]{beamer}
\usepackage[utf8]{inputenc}
\usepackage[T1]{fontenc}
\usepackage{graphicx}
\usepackage{fontspec}
\usepackage{xcolor}
\usepackage{tcolorbox}
\usepackage{listings} % Add the listings package
\usepackage{hyperref} % Add the hyperref package

\lstdefinelanguage{JavaScript}{
    keywords={function, var, let, const, if, else, for, while, return, true, false},
    keywordstyle=\color{blue}\bfseries,
    ndkeywords={class, export, boolean, throw, implements, import, this},
    ndkeywordstyle=\color{orange}\bfseries,
    identifierstyle=\color{black},
    sensitive=false,
    comment=[l]{//},
    morecomment=[s]{/*}{*/},
    commentstyle=\color{gray}\ttfamily,
    stringstyle=\color{green}\ttfamily,
}

\lstset{
    breaklines=true,
    language=JavaScript,
    % ... (other style settings)
}

\lstdefinelanguage{PHP}{
    keywords={class, function, echo, if, else, foreach, for, while, return},
    keywordstyle=\color{blue}\bfseries,
    ndkeywords={public, private, protected, static},
    ndkeywordstyle=\color{purple}\bfseries,
    identifierstyle=\color{black},
    sensitive=false,
    comment=[l]{//},
    morecomment=[s]{/*}{*/},
    commentstyle=\color{gray}\ttfamily,
    stringstyle=\color{green}\ttfamily,
}

\lstset{
    breaklines=true,
    language=PHP,
    % ... (other style settings)
}

\setsansfont[
  ItalicFont=fonts/TitilliumWeb-Italic.ttf,
  BoldFont=fonts/TitilliumWeb-Bold.ttf,
  BoldItalicFont=fonts/TitilliumWeb-BoldItalic.ttf,
]{TitilliumWeb-Regular.ttf}

\subtitle{IF120203-Web Programming}
\title{\Huge {\textbf{09: \\Template Engine}}}
\date[Serial]{\scriptsize {PRU/SPMI/FR-BM-18/0222}}
\author[Pradita]{\small {\textbf{PRADITA UNIVERSITY}}}

\usetheme{Pradita}

\begin{document}
\begin{frame}
    \titlepage
\end{frame}

\begin{frame}{Goals - Using Template Engine}
\begin{itemize}
	\item Understand the concept of template engines in web development
	\item Learn how to integrate and use a template engine in a web application
	\item Explore the benefits of using a template engine for managing dynamic content and improving code maintainability
	\item Apply a template engine to enhance the appearance and functionality of a web application developed in previous sessions
\end{itemize}
\end{frame}

\begin{frame}{Introduction to Blade}
\vskip1cm
\begin{itemize}
	\item Blade is the default template engine in Laravel, a powerful PHP web framework.
	\item It provides a convenient way to work with views and generate dynamic content in web applications.
	\item Blade templates are essentially PHP files with added features that make working with HTML and PHP code more efficient.
	\item Blade syntax is designed to be easy to read and write, enhancing the separation of concerns between logic and presentation.
	\item Blade templates support features like template inheritance, components, conditionals, loops, and more.
	\item The template engine automatically escapes data, providing security against cross-site scripting (XSS) attacks.
\end{itemize}
\end{frame}

\begin{frame}[fragile]{Example Of Using Blade Part 1}
    \begin{lstlisting}[language=PHP]
    public function index()
    {
        $data['title'] = 'Currency Converter';
        $data['currencies'] = [
            'IDR' => 'Indonesia Rupiah',
            'USD' => 'United States Dollar',
            'EUR' => 'Euro Member Countries',
            'GBP' => 'United Kingdom Pound',
            // Add more currencies as needed
        ];
        return view('converter', $data);
    }
    \end{lstlisting}
\end{frame}

\begin{frame}[fragile]{Example Of Using Blade Part 2}
    \begin{itemize}
        \item in \texttt{converter.blade.php} line 18: \verb|<h1>{{ $title }}</h1>|
        \item In line 28-30:
\begin{lstlisting}[language=PHP]
	@foreach($currencies as $code => $name)
                <option value="{{ $code }}">{{ $code }} - {{ $name }}</option>
          @endforeach
\end{lstlisting}
        \item It parses from the \texttt{ConverterController} class index function using \verb|$data|
    \end{itemize}
\end{frame}

\begin{frame}[fragile]{Challenge (More Advance)}
    \begin{itemize}
        \item Seperate the \texttt{converter.blade.php} to 3 parts
        \item header, main content, and footer
        \item Reference: 
        \begin{verbatim}
        https://laravel.com/docs/10.x/blade#extending-a-layout
        \end{verbatim}
    \end{itemize}
\end{frame}

\begin{frame4}
    \frametitle{Thank You}
\end{frame4}

\end{document}
